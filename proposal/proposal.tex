\documentclass{proc}
\usepackage[colorlinks=true]{hyperref}

\begin{document}
    
\title{Generating Street Maps with Tensor Fields}
\author{William Clarkson, Marcella Hastings, Nathaniel Tenczar}

\maketitle

\section{Problem Statement}
Our goal is to generate realistic street maps which based on local geography
such as water boundaries, and which can be easily shaped by user-defined
control points to quickly and easily achieve a desired layout. This has been
achieved in \cite{wonka}. However, their implementation is roughly 100,000
lines of C++, which we found difficult to understand due to a lack
of modularity. We propose that by implementing the core functionality of this
pipeline in a clear and modular way, we will enable future contributors
to comprehend and build more easily upon this work.

\section{Context}
We will implement this pipeline in Haskell and make full use of the functional
programming paradigm to accomplish the goals stated above. As many of the
components of the pipeline are defined mathematically, a functional
language like Haskell will lend itself well to expressing them.

\section{Overview of Solution}
As described in \cite{wonka}, we will generate a tensor field based on a set
of constraints (including water boundaries and user-defined control points
indicating street orientation). We will calculate the major and minor
eigenvectors of this tensor field, and trace hyperstreamlines based on
the eigenvectors to determine the locations of streets. We will then calculate
intersections between these hyperstreamlines to build a graph of streets. We
will then generate a visual representation of this street layout.

\section{Obstacles}
At present, we are unfamiliar with tensor fields and their associated
operations. We identify this as the primary challenge to the completion of this
project. However, we are confident that, with our background knowledge of
mathematics and the wealth of available resources on the topic, we will be able
to implement the necessary representations and operations of tensor fields to
solve the problem as a whole.

\section{Definition of Success}
Our project will be a success if our implementation of the work of Chen et al.
allows a user to provide simple constraints from which an image of a realistic
street layout is generated which is influenced by the provided constraints. If
time permits, we would like to provide an interactive interface in which users
can intuitively add and edit constraints in real time.

\begin{thebibliography}{9}
\bibitem{wonka}
    Chen, G., Esch, G., Wonka, P., Müller, P., Zhang, E.. 2008. \href{http://www.google.com/url?q=http%3A%2F%2Fpeterwonka.net%2FPublications%2Fpdfs%2F2008.SG.Chen.InteractiveProceduralStreetModeling.pdf&sa=D&sntz=1&usg=AFQjCNGp-DPZdfTyZLHmKVeefan3jltzzw}{Interactive Procedural Street Modeling}. ACM Transactions on Graphics.
\bibitem{tensorviz}
    Komura, Taku.  \href{http://www.google.com/url?q=http%3A%2F%2Fwww.inf.ed.ac.uk%2Fteaching%2Fcourses%2Fvis%2Flecture_notes%2Flecture14.pdf&sa=D&sntz=1&usg=AFQjCNGytegkIGcZJd_JM-3pzlilzPVH7g}{Tensor Visualization Lecture Notes}. Institute for Perception, Action \& Behaviour, School of Informatics, University of Edinburgh.
\bibitem{tensors}
    Delmarcelle, T.; Hesselink, L. 2006. \href{http://www.google.com/url?q=http%3A%2F%2Fwww.inf.ethz.ch%2Fpersonal%2Fpeikert%2FSciVis%2FLiterature%2FDelmarcelleHesselink94.pdf&sa=D&sntz=1&usg=AFQjCNGIb52UifdoK_NB0oB8sU_r_rebBQ}{The Topology of Symmetric, Second-Order Tensor Fields}. Visualization and Processing of Tensor Fields.

\end{thebibliography}

\end{document}
